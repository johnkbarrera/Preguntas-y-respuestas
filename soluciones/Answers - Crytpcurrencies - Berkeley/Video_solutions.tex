\documentclass[a4paper,9pt]{article}
\input{prmb}

\usepackage[T1]{fontenc}
\usepackage{fancyvrb}
\usepackage{color}

\begin{document}
\begin{table}[h]
\begin{tabular}{ll}
{\textbf{Videos: Solutions}}&\\
\end{tabular}
\end{table}

\noindent\rule{16cm}{0.5pt}
~\\
\begin{center}
{\huge Bitcoin and Cryptocurrency Technologies}
\end{center}
~\\
~\\
\noindent\rule{16cm}{0.5pt}
	
	\vspace{1cm}
 \textbf{CHAPTER 1: INTRODUCTION TO CRYPTO AND CRYPTOCURRECIES}
		\begin{enumerate}
		
			\vspace{5mm} %5mm vertical space
			\item\textbf{Cryptographic hash functions}\\ \\
			Which of the following is true of SHA-256?
			\begin{enumerate}
				\item It has been proven not to have a collision
				\item We hope that there are no collations
				\item \textcolor[rgb]{0.2,0.8,0.2}{No collision has ever been found}
				\item It has been proven that there is not fast way to find collisions
			\end{enumerate}
			
		  \vspace{5mm} %5mm vertical space
			\item\textbf{Hash pointer and data structures}\\ \\
			Which of the following types of modification of a block chain data structure can be detected? 
			\begin{enumerate}
				\item \textcolor[rgb]{0.2,0.8,0.2}{Insertion of a block}
				\item \textcolor[rgb]{0.2,0.8,0.2}{Deletion of a block}
				\item \textcolor[rgb]{0.2,0.8,0.2}{Tampering of data in a block}
				\item \textcolor[rgb]{0.2,0.8,0.2}{Re-ordering of block }
			\end{enumerate}
			
			\vspace{5mm} %5mm vertical space
			\item\textbf{Digital signatures}\\ \\
			Which of these keys are required for verifying a signature?
			\begin{enumerate}
				\item The secret key
				\item \textcolor[rgb]{0.2,0.8,0.2}{The public key}
				\item Both the secret key and the public key
				\item Node. Keys are required only for signing; anyone can verify the signature without a key
			\end{enumerate}
			
			\vspace{5mm} %5mm vertical space
			\item\textbf{Public key as identities}\\ \\
			If you generate numerous identities (public keys) for yourself and interact on line using those different identities, what might happen?
			\begin{enumerate}
				\item \textcolor[rgb]{0.2,0.8,0.2}{Others might be able to take over your identities if your randomness is bad}
				\item Others may be able to link you identities because public keys generated on the same computer look similar
				\item \textcolor[rgb]{0.2,0.8,0.2}{Others may be able to de-anonymize you by analyzing your activity patterns}
			\end{enumerate}
			
			\vspace{5mm} %5mm vertical space
			\item\textbf{A simple Cryptocurrency}\\ \\
			In Scrooge Coin, you have ten coins each of value 3.0. You'd like to transfer coins of value 5.0 to your friends. This require.
			\begin{enumerate}
				\item One transaction, one new coin created, and one signature
				\item \textcolor[rgb]{0.2,0.8,0.2}{One transaction, two new coins created. And two signatures}
				\item Two transaction, two new coins created, and four signatures
				\item Two transaction, one new coins created, and two signature
			\end{enumerate}
			
		\end{enumerate}

%%%%%%%%%%%%%%%%%%%%%%%%%%%%%%%%%%%%%%%%%%%%%%%%%%%%%%%%%%%%%%%%%%%%%%%%%%%%%%%%%%%%%%%%%%%%%%%%%%%%%%%%%%%%%%%%%%%%%%%%%%%%%%%%%%%%%%%%%%%%%%%%%%%%%%%%%%%%%%%%%		
	
	\vspace{1cm}
	\textbf{CHAPTER 2: HOW BITCOIN ACHIEVES DECENTRALIZATION  }
		\begin{enumerate}
		
			\vspace{5mm} %5mm vertical space
			\item\textbf{Centralization vs. Decentralization}\\ \\
			Which of these factors make distributes consensus hard?
			\begin{enumerate}
				\item \textcolor[rgb]{0.2,0.8,0.2}{Nodes may crash}
				\item \textcolor[rgb]{0.2,0.8,0.2}{Nodes maybe be taken over by malware}
				\item Encrypted message may be intercepted and decrypted
				\item \textcolor[rgb]{0.2,0.8,0.2}{There is latency on the network}
			\end{enumerate}
			
		  \vspace{5mm} %5mm vertical space
			\item\textbf{Distributed consensus}\\ \\
			Why is bitcoin able to reach consensus in practice despite this being a generally difficult problem?
			\begin{enumerate}
				\item \textcolor[rgb]{0.2,0.8,0.2}{Financial incentives cause participant work together}
				\item Only small groups of nodes have to reach consensus rather than the network having to globally reach consensus
				\item The order of blocks doesn't matter for consensus
				\item \textcolor[rgb]{0.2,0.8,0.2}{Consensus only has to be reached over long time scales}
			\end{enumerate}
			
			\vspace{5mm} %5mm vertical space
			\item\textbf{Consensus without identities: the block chain}\\ \\
			What can a malicious node do?
			\begin{enumerate}
				\item Create valid transactions originating from someone else address
				\item Prevent a valid transaction from getting any confirmations
				\item \textcolor[rgb]{0.2,0.8,0.2}{Ignore the longest valid branch rule when proposing a new block}
			\end{enumerate}
			
			\vspace{5mm} %5mm vertical space
			\item\textbf{Incentives of proof of work}\\ \\
			Proof of work is a way to:
			\begin{enumerate}
				\item \textcolor[rgb]{0.2,0.8,0.2}{Select nodes in proportion of computing power}
				\item \textcolor[rgb]{0.2,0.8,0.2}{Let nodes compete for the `right' to creates blocks}
				\item Make it impossible for one miner to act like many different miners
			\end{enumerate}
			\vspace{3mm} %5mm vertical space
			A block in the block chain was found at time t. what is the probability that next block was found at or before t + 10 minutes? Assume that the total hash power stay constant. 
			\begin{enumerate}
				\item \textcolor[rgb]{0.2,0.8,0.2}{More than 50\%}
				\item Less than 50\%
				\item Exactly 50\%
			\end{enumerate}
			
			\vspace{5mm} %5mm vertical space
			\item\textbf{Putting it all together}\\ \\
			A 51\% attacker can potentially:
			\begin{enumerate}
				\item Steal coin from an existing address
				\item \textcolor[rgb]{0.2,0.8,0.2}{Make it unprofitable for other miners to mine}
				\item Change the block reward 
				\item \textcolor[rgb]{0.2,0.8,0.2}{Suppress transactions form the block chain}
			\end{enumerate}	
			\vspace{3mm} %5mm vertical space
			Which of the following are true?
			\begin{enumerate}
				\item 51\% attacks are difficulty because an adversary would need to control more than half of the nodes on the bitcoin network
				\item \textcolor[rgb]{0.2,0.8,0.2}{Proof-of-work is essential for preventing Sybil attacks on the bitcoin blockchain}
				\item \textcolor[rgb]{0.2,0.8,0.2}{As a transaction gets buried deeper in the blockchain, it becomes less and less likely that it will ever be undone because the work require to make a longer alternate branch becomes more and more difficult }
			\end{enumerate}
	
		\end{enumerate}	

%%%%%%%%%%%%%%%%%%%%%%%%%%%%%%%%%%%%%%%%%%%%%%%%%%%%%%%%%%%%%%%%%%%%%%%%%%%%%%%%%%%%%%%%%%%%%%%%%%%%%%%%%%%%%%%%%%%%%%%%%%%%%%%%%%%%%%%%%%%%%%%%%%%%%%%%%%%%%%%%%
			
	\vspace{1cm}
	\textbf{CHAPTER 3: MECHANICS OF BITCOIN}
		\begin{enumerate}
		
			\vspace{5mm} %5mm vertical space
			\item\textbf{Bitcoin transactions}\\ \\
			In a typical transaction
			\begin{enumerate}
				\item There is one signature that covers all the inputs
				\item \textcolor[rgb]{0.2,0.8,0.2}{Each input contains a signature}
				\item There is one signature that covers all the output
				\item Each output contains a signature
			\end{enumerate}
			
		  \vspace{5mm} %5mm vertical space
			\item\textbf{Bitcoins Scripts}\\ \\
			Bitcoin's script supports instructions whose effect is:
			\begin{enumerate}
				\item \textcolor[rgb]{0.2,0.8,0.2}{Adding two numbers}
				\item \textcolor[rgb]{0.2,0.8,0.2}{Conditions execution (if/then)}
				\item Looping
				\item Recursion
				\item \textcolor[rgb]{0.2,0.8,0.2}{Hashing}
			\end{enumerate}
			
			\vspace{5mm} %5mm vertical space
			\item\textbf{Applications of bitcoin scripts}\\ \\
			Alice is paying for a service using bitcoin micropayments. If she simply disconnects at some point without notifying Bob and stops sending micropayments. What can Bob do?
			\begin{enumerate}
				\item Bob is out of luck. He doesn't earn any bitcoin and must pursue legal recourse
				\item Bob can redeem the maximum amount that Alice initially escrowed into a multisig address
				\item \textcolor[rgb]{0.2,0.8,0.2}{Bob can redeem the latest micropayment transaction that Alice sent in the last time period before disconnecting, which matches the length of services she received.}
				\item Bob can refuse to sign the refund transaction, so both Alice and Bob will end up losing Bitcoins, which will sit in the multisig escrow forever
			\end{enumerate}
			\vspace{3mm} %5mm vertical space
			Bitcoin micropayments require the use of:
			\begin{enumerate}
				\item \textcolor[rgb]{0.2,0.8,0.2}{Multisignature transactions}
				\item Proof of burn
				\item \textcolor[rgb]{0.2,0.8,0.2}{Time-locked transactions}
				\item Pay-to-script-hash
			\end{enumerate}
			
			\vspace{5mm} %5mm vertical space
			\item\textbf{Bitcoin blocks}\\ \\
			Blocks contain a tree of transactions instead of a flat list because:
			\begin{enumerate}
				\item It result is smaller blocks
				\item It�s easier to insert or delete new transactions while the block is being assembles 
				\item \textcolor[rgb]{0.2,0.8,0.2}{It enables efficiently proving that a transaction is include in a block}
			\end{enumerate}
			
			\vspace{5mm} %5mm vertical space
			\item\textbf{The bitcoin network}\\ \\
			If two transactions A-->B and A<--C are both broadcasted almost simultaneously from different modes, what determines which one will eventually end up in the block chain?
			\begin{enumerate}
				\item The transaction that reaches the majority of nodes first will win
				\item The transactions that was broadcast first will win
				\item \textcolor[rgb]{0.2,0.8,0.2}{The miner who finds the next block will likely resolve the tie by including one of the transactions in the block}
				\item Each node has its own version of the block chain containing the transaction that it heard about first 
			\end{enumerate}
			
			\vspace{5mm} %5mm vertical space
			\item\textbf{Limitations and improvements}\\ \\
			Which of the following requires a hard fork?
			\begin{enumerate}
				\item Disabling the OP\_SHA1 instruction
				\item A requirement that each transaction have it outputs sorted by values in ascending (or not-decreasing) order
				\item \textcolor[rgb]{0.2,0.8,0.2}{Increasing the maximum permitted size of blocks}
				\item Decreasing the maximum permitted size of blocks
				\item \textcolor[rgb]{0.2,0.8,0.2}{Adding a new OP\_SHA3 script instruction}
			\end{enumerate}
	
		\end{enumerate}

%%%%%%%%%%%%%%%%%%%%%%%%%%%%%%%%%%%%%%%%%%%%%%%%%%%%%%%%%%%%%%%%%%%%%%%%%%%%%%%%%%%%%%%%%%%%%%%%%%%%%%%%%%%%%%%%%%%%%%%%%%%%%%%%%%%%%%%%%%%%%%%%%%%%%%%%%%%%%%%%%
	
	\vspace{1cm}
	\textbf{CHAPTER 4: HOW TO STORE AND USE BITCOINS}
		\begin{enumerate}
		
			\vspace{5mm} %5mm vertical space
			\item\textbf{How to store and use bitcoins}\\ \\
			What is bitcoin wallet?
			\begin{enumerate}
				\item An address that contains a lot of unspent bitcoins
				\item \textcolor[rgb]{0.2,0.8,0.2}{A piece of software that remember an individual's bitcoins address and keys}
				\item A type of mining software
				\item An online exchange that people can go to in order to acquire bitcoins
			\end{enumerate}
			
		  \vspace{5mm} %5mm vertical space
			\item\textbf{Hot and cold storage}\\ \\
			Which of the following statements are true about cold wallet storage? (check all that apply) 
			\begin{enumerate}
				\item \textcolor[rgb]{0.2,0.8,0.2}{Cold storage stores keys in a device without network access}
				\item Cold storage tends to be more convenient
				\item Cold storage can store more bitcoins
				\item \textcolor[rgb]{0.2,0.8,0.2}{Hot storage wallets can generate arbitrarily many cold storage addresses without contacting the cold storage}
			\end{enumerate}
			
			\vspace{5mm} %5mm vertical space
			\item\textbf{Splitting and sharing keys}\\ \\
			In the K-out-of-N secret sharing scheme presented, the size of each share (in bits) will be
			\begin{enumerate}
				\item 1/K times the size of the secret	
				\item \textcolor[rgb]{0.2,0.8,0.2}{Equal to the size of the secret}
				\item K times the size of the secret 
				\item N times the size of the secret	
			\end{enumerate}
			
			\vspace{5mm} %5mm vertical space
			\item\textbf{Online wallets and exchanges}\\ \\
			Which of these are risks of Bitcoin exchanges that are NOT risks of maintaining one's own hot or cold wallet? (check all that apply) 
			\begin{enumerate}
				\item \textcolor[rgb]{0.2,0.8,0.2}{Bank runs}
				\item \textcolor[rgb]{0.2,0.8,0.2}{Ponzi schemes}
				\item Key compromises or leaks	
				\item Double-spend attacks
			\end{enumerate}
			
			\vspace{5mm} %5mm vertical space
			\item\textbf{Payment services}\\ \\
			In the scenario presented, which of these parties are exposed to exchange rate risk? (check all that apply) 
			\begin{enumerate}
				\item \textcolor[rgb]{0.2,0.8,0.2}{User}
				\item Merchant \\ \textcolor[rgb]{0.49,0.62,0.75}{The user is exposed to exchange rate rick to some degree since they must hold bitcoins at least temporally to be able to pay whit bitcoins.}
				\item \textcolor[rgb]{0.2,0.8,0.2}{Payment service}
			\end{enumerate}
			
			\vspace{5mm} %5mm vertical space
			\item\textbf{Transactions fees}\\ \\
			Doesn't have questions 
			
			\vspace{5mm} %5mm vertical space
			\item\textbf{Currency exchange markets}\\ \\
			In the model presented, which of these are sources of demand for bitcoins? (check all that apply) 
			\begin{enumerate}
				\item \textcolor[rgb]{0.2,0.8,0.2}{Mediating fiat-currency transactions} \\ \textcolor[rgb]{0.49,0.62,0.75}{The model presented considered only two sources of demand: meadiating fiat-currency transaction and investment. Theoretical model must often leave out some factors that matter in practice in order to keep the analysis tractable. Arguably, paying transaction fees is also a source of demand for bitcoins.}
				\item Demand deposits of bitcoins
				\item Gambling 
				\item \textcolor[rgb]{0.2,0.8,0.2}{Investment} \\ \textcolor[rgb]{0.49,0.62,0.75}{The model presented considered only two sources of demand: meadiating fiat-currency transaction and investment. Theoretical model must often leave out some factors that matter in practice in order to keep the analysis tractable. Arguably, paying transaction fees is also a source of demand for bitcoins.}
				\item Paying transaction fees
			\end{enumerate}		
		\end{enumerate}
		
%%%%%%%%%%%%%%%%%%%%%%%%%%%%%%%%%%%%%%%%%%%%%%%%%%%%%%%%%%%%%%%%%%%%%%%%%%%%%%%%%%%%%%%%%%%%%%%%%%%%%%%%%%%%%%%%%%%%%%%%%%%%%%%%%%%%%%%%%%%%%%%%%%%%%%%%%%%%%%%%%%

\vspace{2cm}
	\textbf{CHAPTER 5: BITCOIN MINING}
		\begin{enumerate}
		
			\vspace{5mm} %5mm vertical space
			\item\textbf{The task of Bitcoin Miners}\\ \\
			Which of the following are true about Bitcoin miners? 
			\begin{enumerate}
				\item \textcolor[rgb]{0.2,0.8,0.2}{The target hash has become so small that the block header nonce alone isn't generally large enough to allow miners so search enough of the hash output space to find a valid block}
				\item Bitcoin miners can more efficiently mine for blocks by specifically targeting parts of the nonce search space that have more puzzle solutions
				\item Over a 2 week period, the average time to mine a block is always 10 minutes
				\item \textcolor[rgb]{0.2,0.8,0.2}{The mining difficulty is recomputed roughly every 2 weeks to keep the proof-of-work puzzle difficult}
			\end{enumerate}
			
		  \vspace{5mm} %5mm vertical space
			\item\textbf{Mining Hardware}\\ \\
			Which statement about Bitcoin miners is NOT true?  
			\begin{enumerate}
				\item If the global hash rate doubles every two months, a new piece of hardware that a miner buys will find most of the blocks that it ever will mine in the first six months of operation
				\item \textcolor[rgb]{0.2,0.8,0.2}{Bitcoin miners can recoup a reasonable fraction of their initial expenses by selling their ASICs once they are done with them to other users for less computationally intense purposes}
				\item Many miners will consider the climate of an area when setting up mining operations because of the cost of cooling their equipment
				\item Mining Bitcoin on a modern CPU will yield negligible mining rewards	
			\end{enumerate}
			
			\vspace{5mm} %5mm vertical space
			\item\textbf{Energy Consumption \& Ecology}\\ \\
			Which of the following are assumptions made about the UPPER bound for the energy used for mining Bitcoins? (check all that apply) 
			\begin{enumerate}
				\item Everyone mines where it is cold (cooling doesn't consume energy)
				\item Everyone mines at the maximum claimed efficiency	
				\item \textcolor[rgb]{0.2,0.8,0.2}{Miners mine up to the point that all of the money they earn is used to pay for electricity}
				\item The energy efficiency of mining hardware decreases with age
				\item \textcolor[rgb]{0.2,0.8,0.2}{Miners all pay the same for electricity}
			\end{enumerate}
		 \vspace{3mm} %5mm vertical space
			Which of the following are assumptions made about the LOWER bound for the energy used for mining Bitcoins? (check all that apply) 
			\begin{enumerate}
				\item \textcolor[rgb]{0.2,0.8,0.2}{Everyone mines where it is cold (cooling doesn't consume energy) }
				\item \textcolor[rgb]{0.2,0.8,0.2}{Everyone mines at the maximum claimed efficiency}
				\item Miners mine up to the point that all of the money they earn is used to pay for electricity
				\item The energy efficiency of mining hardware decreases with age
			\end{enumerate}
		
		
			\vspace{5mm} %5mm vertical space
			\item\textbf{Mining Pools}\\ \\
			Mining pools... 
			\begin{enumerate}
				\item Let members earn more rewards, on average, than they would by mining alone 
				\item \textcolor[rgb]{0.2,0.8,0.2}{Typically make all their members search for blocks with the same coinbase address (the address that receives mining rewards) }
				\item Evenly divide up block rewards between all members of the pool
				\item Can undermine the security of Bitcoins consensus algorithm, but this isn't a problem in practice since the majority of miners aren't part of pools
			\end{enumerate}
			
			\vspace{5mm} %5mm vertical space
			\item\textbf{Mining Incentives and Strategies}\\ \\
			Which of the following are true about potential mining strategies Bitcoin miners can employ? (check all that apply) 
			\begin{enumerate}
				\item \textcolor[rgb]{0.2,0.8,0.2}{Miners who control more mining power have more potentially profitable strategies available}
				\item Block withholding, forking, and other attacks have been frequently carried out in practice
				\item \textcolor[rgb]{0.2,0.8,0.2}{Some alternative strategies might be motivated by goals other than earning more bitcoins}
			\end{enumerate}
			
		\end{enumerate}
				
%%%%%%%%%%%%%%%%%%%%%%%%%%%%%%%%%%%%%%%%%%%%%%%%%%%%%%%%%%%%%%%%%%%%%%%%%%%%%%%%%%%%%%%%%%%%%%%%%%%%%%%%%%%%%%%%%%%%%%%%%%%%%%%%%%%%%%%%%%%%%%%%%%%%%%%%%%%%%%%%%		
\vspace{1.5cm}
	\textbf{CHAPTER 6: BITCOIN AND ANONYMITY}
		\begin{enumerate}
		
			\vspace{5mm} %5mm vertical space
			\item\textbf{Anonymity Basics}\\ \\
			Unlinkability in Bitcoin could mean
			\begin{enumerate}
				\item \textcolor[rgb]{0.2,0.8,0.2}{It's hard to link different address  owned by the same user}
				\item \textcolor[rgb]{0.2,0.8,0.2}{It's hard to link different transactions made by the same user}
				\item It's hard to link different transactions having the same output address
			\end{enumerate}
			
		  \vspace{5mm} %5mm vertical space
			\item\textbf{How to de-anomymize Bitcoin}\\ \\
			Which of the following observations  would suggest that address A and B may be controlled by the same user/entity
			\begin{enumerate}
				\item There is a transaction with A as a input address and B as output addresses 
				\item \textcolor[rgb]{0.2,0.8,0.2}{There is a transaction with both A and B as input addresses} 
				\item There is a transaction with both A and B as output addresses 
			\end{enumerate}
			
			\vspace{5mm} %5mm vertical space
			\item\textbf{Mixing}\\ \\
			Which of these techniques can improve the anonymity provide by mixing services?(check all that apply)   
			\begin{enumerate}
				\item \textcolor[rgb]{0.2,0.8,0.2}{Using a series of mixes}
				\item \textcolor[rgb]{0.2,0.8,0.2}{Using the same `chunk' size for all mixing transactions}
				\item Charging a constant percentage of the transaction value as a mixing fee
			\end{enumerate}
		
			\vspace{5mm} %5mm vertical space
			\item\textbf{Decentralized Mixing}\\ \\
			Which of these is NOT an advantage of CoinJoin over centralized mixes
			\begin{enumerate}
				\item There by mixes is impossible
				\item \textcolor[rgb]{0.2,0.8,0.2}{Built-in protection against denial-of-service attacks}
				\item Potentially better anonymity because centralized mines can e compromised by adversaries. 
			\end{enumerate}
			
			\vspace{5mm} %5mm vertical space
			\item\textbf{Zerocoin and Zerocash}\\ \\
			What is `zero knowledge' about zero-knowledge proof?
			\begin{enumerate}
				\item It take no outside knowledge to verify the correctness of the proof
				\item \textcolor[rgb]{0.2,0.8,0.2}{It is proof that doesn't reveal any knowledge to create}
				\item It is a proof that requires no knowledge to create
				\item It is a proof that you have no knowledge of something 		
			\end{enumerate}
			
			\vspace{5mm} %5mm vertical space
			\item\textbf{Tor and the Silk Road}\\ \\
			Doesn't have questions 
						
		\end{enumerate}
		
%%%%%%%%%%%%%%%%%%%%%%%%%%%%%%%%%%%%%%%%%%%%%%%%%%%%%%%%%%%%%%%%%%%%%%%%%%%%%%%%%%%%%%%%%%%%%%%%%%%%%%%%%%%%%%%%%%%%%%%%%%%%%%%%%%%%%%%%%%%%%%%%%%%%%%%%%%%%%%%%%
\vspace{1.0cm}
	\textbf{CHAPTER 7: COMMUNITY, POLITICS, AND REGULATION}
		\begin{enumerate}
		
			\vspace{5mm} %5mm vertical space
			\item\textbf{Consensus in Bitcoin}\\ \\
			Doesn't have questions 
			
		  \vspace{5mm} %5mm vertical space
			\item\textbf{Bitcoin Core Software}\\ \\
			Which of the following is true about a fork of Bitcoin's rules which results in a fork of the block chain into two branches? 
			\begin{enumerate}
				\item Anyone who owned bitcoins before the fork can choose which branch to transfer their coins to
				\item The fork will eventually be resolved due to the longest valid branch rule
				\item \textcolor[rgb]{0.2,0.8,0.2}{The fork doubles the total value of the currency}
				\item A transaction can be valid in both forks
			\end{enumerate}
			
			\vspace{5mm} %5mm vertical space
			\item\textbf{stakeholders: Who's in charge?}\\ \\
			Which participants in the Bitcoin ecosystem have some amount of power in a negotiation about rule-setting?   
			\begin{enumerate}
				\item \textcolor[rgb]{0.2,0.8,0.2}{Bitcoin Core developers}
				\item \textcolor[rgb]{0.2,0.8,0.2}{Miners}
				\item \textcolor[rgb]{0.2,0.8,0.2}{Investors}	
				\item \textcolor[rgb]{0.2,0.8,0.2}{Merchants}	
				\item \textcolor[rgb]{0.2,0.8,0.2}{Payment services}
			\end{enumerate}
		
			\vspace{5mm} %5mm vertical space
			\item\textbf{Roots of Bitcoin}\\ \\
			Doesn't have questions 
			
			\vspace{5mm} %5mm vertical space
			\item\textbf{Governments Notice Bitcoin}\\ \\
			According to the lecture, what's one way that governments have tried to enforce capital controls in a world with Bitcoin? 
			\begin{enumerate}
				\item Shutting down Bitcoin-based markets for illegal items such as Silk Road
				\item \textcolor[rgb]{0.2,0.8,0.2}{Disconnecting Bitcoin from the local fiat currency }
				\item Blocking the Bitcoin protocol
				\item Purchasing mining hardware and attempting a Goldfinger attack
			\end{enumerate}
			
			\vspace{5mm} %5mm vertical space
			\item\textbf{Anti Money-Laundering}\\ \\
			According to the lecture, what steps do governments take to prevent money laundering? (Check all that apply) 
			\begin{enumerate}
				\item \textcolor[rgb]{0.2,0.8,0.2}{Require some businesses that handle money to know their customers identities}
				\item Constantly monitor the Bitcoin network and block chain 
				\item \textcolor[rgb]{0.2,0.8,0.2}{Require a variety of companies to file reports describing any large transactions they are a party to}
				\item Limit the maximum size of financial transactions
			\end{enumerate}
			
			\vspace{5mm} %5mm vertical space
			\item\textbf{Regulation}\\ \\
			A reputation-based approach to fixing a lemons market might not work:
			\begin{enumerate}
				\item \textcolor[rgb]{0.2,0.8,0.2}{At the end of a seller's presence in the market}	
				\item When sellers don't provide warranties for products
				\item \textcolor[rgb]{0.2,0.8,0.2}{When consumers don't do repeat business with the same entity}
				\item \textcolor[rgb]{0.2,0.8,0.2}{At the beginning of a sellers presence in the market}
			\end{enumerate}
			\vspace{3mm} %5mm vertical space
			Which of these are signs that there might be a market failure?
			\begin{enumerate}
				\item \textcolor[rgb]{0.2,0.8,0.2}{The market is completely unregulated}
				\item Sellers agree with each other to raise prices
				\item Sellers agree not to compete with each other and offer a reduced selection of products
			\end{enumerate}
			
			\vspace{5mm} %5mm vertical space
			\item\textbf{New York's BitLicense Proposal}\\ \\
			Doesn't have questions 
			
		\end{enumerate}
		
%%%%%%%%%%%%%%%%%%%%%%%%%%%%%%%%%%%%%%%%%%%%%%%%%%%%%%%%%%%%%%%%%%%%%%%%%%%%%%%%%%%%%%%%%%%%%%%%%%%%%%%%%%%%%%%%%%%%%%%%%%%%%%%%%%%%%%%%%%%%%%%%%%%%%%%%%%%%%%%%%
\vspace{1.0cm}
	\textbf{CHAPTER 8: ALTERNATIVE MINING PUZZLES}
		\begin{enumerate}
		
			\vspace{5mm} %5mm vertical space
			\item\textbf{Essential Puzzle Requirements}\\ \\
 			Doesn't have questions 
			
		  \vspace{5mm} %5mm vertical space
			\item\textbf{ASIC Resistant Puzzles}\\ \\
			ASIC resistance... 
			\begin{enumerate}
				\item \textcolor[rgb]{0.2,0.8,0.2}{seeks to make it more appealing to mine with regular consumer devices than it is today}
				\item has been successfully achieved in practice using the `script' memory-hard hash function
				\item \textcolor[rgb]{0.2,0.8,0.2}{is a response to the centralization of Bitcoin mining}
			\end{enumerate}
			
			\vspace{5mm} %5mm vertical space
			\item\textbf{Proof-of-usefol-work}\\ \\
			Proof-of-useful-work cryptocurrency designs... 
			\begin{enumerate}
				\item \textcolor[rgb]{0.2,0.8,0.2}{differ from traditional volunteer distributed computing projects because cryptocurrencies cannot rely on a trusted administrator to select and distribute the problems to be solved}
				\item have been successfully used to solve computational problems such as protein folding 
				\item \textcolor[rgb]{0.2,0.8,0.2}{should preferably be based on problems whose solution benefits the public, rather than the solver, to avoid skewing the incentives of miners}
			\end{enumerate}
		
			\vspace{5mm} %5mm vertical space
			\item\textbf{Nonoutsourceable Puzzle}\\ \\
			In a vigilante attack against a mining pool, the attacker,
			\begin{enumerate}
				\item discards both shares and blocks that he finds
				\item \textcolor[rgb]{0.2,0.8,0.2}{submits shares but discards blocks}
				\item discards shares while submitting blocks to a different mining pool
			\end{enumerate}	
			
			\vspace{5mm} %5mm vertical space
			\item\textbf{Proof-of-Stake `Virtual Mining'}\\ \\
			Which of these is true of virtual mining? 
			\begin{enumerate}
				\item \textcolor[rgb]{0.2,0.8,0.2}{Virtual mining does away with most of the power requirements of proof-of-work systems}
				\item A proof-of-stake system makes 51\% attacks impossible
				\item \textcolor[rgb]{0.2,0.8,0.2}{Several variations of virtual mining have been proposed}
			\end{enumerate}	
		
		\end{enumerate}	
%%%%%%%%%%%%%%%%%%%%%%%%%%%%%%%%%%%%%%%%%%%%%%%%%%%%%%%%%%%%%%%%%%%%%%%%%%%%%%%%%%%%%%%%%%%%%%%%%%%%%%%%%%%%%%%%%%%%%%%%%%%%%%%%%%%%%%%%%%%%%%%%%%%%%%%%%%%%%%%%%
	
\vspace{1.0cm}
	\textbf{CHAPTER 9: BITCOIN AS PLATFORM}
		\begin{enumerate}
		
			\vspace{5mm} %5mm vertical space
			\item\textbf{Bitcoin as an Append-Only Log}\\ \\
 			In which of these situations could secure timestamping be useful? Assume that there is no way to prove that you didn't timestamp multiple values
			\begin{enumerate}
				\item Proving that you don't know something at a specific time 
				\item \textcolor[rgb]{0.2,0.8,0.2}{Proving possession of a document at a specific time}
				\item Securely saving a document at a specific time 
				\item Proving that you can predict the winner of the 2016 US presidential election
			\end{enumerate}
			
		  \vspace{5mm} %5mm vertical space
			\item\textbf{Bitcoin as Smart Property}\\ \\
			The OpenAssets protocol works by 
			\begin{enumerate}
				\item Enabling conversion between Bitcoin and a new type of coin 
				\item Forking Bitcoin to allow many different types of "colored" coins
				\item \textcolor[rgb]{0.2,0.8,0.2}{Associating extra metadata with bitcoins}
				\item Exploiting non-fungibility of bitcoins to impose a blacklist
			\end{enumerate}
			
			\vspace{5mm} %5mm vertical space
			\item\textbf{Secure Multi-Party Lotteries in Bitcoin}\\ \\
			Which of the following features does the Bitcoin secure multi-party lottery system presented depend on?
			\begin{enumerate}
				\item \textcolor[rgb]{0.2,0.8,0.2}{Hash commitments} 
				\item Colored coins
				\item \textcolor[rgb]{0.2,0.8,0.2}{Multisignatures}
				\item \textcolor[rgb]{0.2,0.8,0.2}{Time-locked transactions}
				\item Micropayments	
			\end{enumerate}
		
			\vspace{5mm} %5mm vertical space
			\item\textbf{Bitcoin as Randomness Source}\\ \\
			Which of these are advantages of using the Bitcoin blockchain to generate a cryptographic beacon?
			\begin{enumerate}
				\item \textcolor[rgb]{0.2,0.8,0.2}{The beacon outputs random bits at frequent intervals }
				\item Fresh random bits can be obtained at any desired future time 
				\item Manipulating the beacon output requires 51% mining power
				\item \textcolor[rgb]{0.2,0.8,0.2}{The cost to an attacker to manipulate the beacon's output is quantifiable }
				\item \textcolor[rgb]{0.2,0.8,0.2}{No central authority is needed}
			\end{enumerate}	
			
			\vspace{5mm} %5mm vertical space
			\item\textbf{Prediction Markets \& Real-World Data Feeds}\\ \\
			In a prediction market where shares pay out if and only if a potential future event happens: 
			\begin{enumerate}
				\item The average price of all shares traded as a fraction of the payout is an estimate of the event's likelihood
				\item \textcolor[rgb]{0.2,0.8,0.2}{The current price of shares as a fraction of the payout is an estimate of the event's likelihood}
				\item \textcolor[rgb]{0.2,0.8,0.2}{A trader who is able to control or influence the outcome of the event will likely be able to make a profit}
			\end{enumerate}	
		
		\end{enumerate}	
%%%%%%%%%%%%%%%%%%%%%%%%%%%%%%%%%%%%%%%%%%%%%%%%%%%%%%%%%%%%%%%%%%%%%%%%%%%%%%%%%%%%%%%%%%%%%%%%%%%%%%%%%%%%%%%%%%%%%%%%%%%%%%%%%%%%%%%%%%%%%%%%%%%%%%%%%%%%%%%%%	
	
\vspace{1.0cm}
	\textbf{CHAPTER 10: ALTCOINS AND THE CRYPTOCURRENCY ECOSYSTEM}
		\begin{enumerate}
		
			\vspace{5mm} %5mm vertical space
			\item\textbf{Short History of Altcoins}\\ \\
 			Which of these statements about altcoins are true?
			\begin{enumerate}
				\item \textcolor[rgb]{0.2,0.8,0.2}{Bitcoin has a higher "market capitalization" than all altcoins combined }
				\item Bitcoin is the most widely forked cryptocurrency
				\item \textcolor[rgb]{0.2,0.8,0.2}{Namecoin supports additional functionality such as domain-name registration that is not found in Bitcoin}
			\end{enumerate}
			
		  \vspace{5mm} %5mm vertical space
			\item\textbf{Interaction Between Bitcoin and Altcoins}\\ \\
			In a merge-mined altcoin scenario:
			\begin{enumerate}
				\item Bitcoin blocks include transactions from the altcoin 
				\item Altcoin block headers include a hash pointer to a Bitcoin block
				\item \textcolor[rgb]{0.2,0.8,0.2}{Bitcoin block headers include the merkle root of transactions for an altcoin block}
				\item The altcoin has the same hash target as Bitcoin at all times
			\end{enumerate}
			
			\vspace{5mm} %5mm vertical space
			\item\textbf{Lifecycle of an Altcoin}\\ \\
 			Doesn't have questions 
		
			\vspace{5mm} %5mm vertical space
			\item\textbf{Bitcoin-Backed Altcoins, `Side Chains'}\\ \\
			Here are several ways in which new coins in an altcoin can be allocated to users. Which of these require changes to Bitcoin? 
			\begin{enumerate}
				\item All coins in the altcoin are generated through (merge) mining and are allocated to miners
				\item Altcoin allocation is "grandfathered" from Bitcoin - every owner of bitcoins becomes the owner of a certain number of altcoins (in a fixed proportion of bitcoins to altcoins)
				\item \textcolor[rgb]{0.2,0.8,0.2}{Bitcoin is used as a "reserve currency" for the altcoin - a unit of the altcoin can be created by putting 1 BTC into escrow; the bitcoin can be released by provably destroying one altcoin unit}
				\item A unit of the altcoin is created by provably destroying one bitcoin
			\end{enumerate}	
			
		\end{enumerate}	
%%%%%%%%%%%%%%%%%%%%%%%%%%%%%%%%%%%%%%%%%%%%%%%%%%%%%%%%%%%%%%%%%%%%%%%%%%%%%%%%%%%%%%%%%%%%%%%%%%%%%%%%%%%%%%%%%%%%%%%%%%%%%%%%%%%%%%%%%%%%%%%%%%%%%%%%%%%
		
\vspace{1.0cm}
	\textbf{CHAPTER 11: THE FUTURE OF BITCOIN}
		\begin{enumerate}
		
			\vspace{5mm} %5mm vertical space
			\item\textbf{The Block Chain as a Vehicle for Decentralization}\\ \\
 			In the smart property scenario presented, where Alice sells her car to Bob via an atomic transaction:
			\begin{enumerate}
				\item Bob must make sure that Alice deletes her private key so that she does not retain the ability to activate the car
				\item Alice and Bob must be physically near the car for the transfer of control to take effect
				\item Requires modifications to Bitcoin because there is no way for two different people who don't trust each other to securely sign the same transaction
				\item \textcolor[rgb]{0.2,0.8,0.2}{The protocol doesn't prove to Bob, before the sale, that the transaction output that Alice wants to sell him actually corresponds to the car he wants to buy}
			\end{enumerate}			
						
		  \vspace{5mm} %5mm vertical space
			\item\textbf{Routes of Blockchain Integration}\\ \\
			Which of these are potential ways to improve security when using Bitcoin as a platform for decentralized commerce?
			\begin{enumerate}
				\item \textcolor[rgb]{0.2,0.8,0.2}{Atomic exchange}
				\item \textcolor[rgb]{0.2,0.8,0.2}{Reputation}
				\item Warranties 
				\item \textcolor[rgb]{0.2,0.8,0.2}{Escrow and dispute mediation}
			\end{enumerate}
			
			\vspace{5mm} %5mm vertical space
			\item\textbf{What Can We Decentralize?}\\ \\
 			Data feeds...
			\begin{enumerate}
				\item \textcolor[rgb]{0.2,0.8,0.2}{allow arbiters to assert facts about the world into the block chain} 
				\item \textcolor[rgb]{0.2,0.8,0.2}{are useful for implementing decentralized prediction markets}
				\item \textcolor[rgb]{0.2,0.8,0.2}{can be decentralized to a degree using Multisignature}
			\end{enumerate}	
		
			\vspace{5mm} %5mm vertical space
			\item\textbf{When is Decentralization a Good Idea}\\ \\
			According to the lecture, what are some issues with using cryptography to enforce contracts? 
			\begin{enumerate}
				\item \textcolor[rgb]{0.2,0.8,0.2}{It is problematic if the state does not recognize a block chain based notion of property because the state will always be the final arbiter}
				\item The technology behind smart contracts is not powerful enough to express the logic needed for real-world contracts like derivatives
				\item \textcolor[rgb]{0.2,0.8,0.2}{Cryptographic security lacks the corrective controls of real-world security such as prosecution of criminals}
				\item \textcolor[rgb]{0.2,0.8,0.2}{Losing a device containing your private keys could result in the inability to use your smart property}
			\end{enumerate}	
			
		\end{enumerate}	
%%%%%%%%%%%%%%%%%%%%%%%%%%%%%%%%%%%%%%%%%%%%%%%%%%%%%%%%%%%%%%%%%%%%%%%%%%%%%%%%%%%%%%%%%%%%%%%%%%%%%%%%%%%%%%%%%%%%%%%%%%%%%%%%%%%%%%%%%%%%%%%%%%%%%%%%%%%

\end{document}

